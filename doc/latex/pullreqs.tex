\documentclass[conference]{IEEEtran}

\begin{document}

\title{An exploration of pull requests on Github}

\author{\IEEEauthorblockN{Georgios Gousios and Martin Pinzger and Arie van
Deursen}
\IEEEauthorblockA{
Software Engineering Research Group\\
Delft University of Technology\\
Delft, The Netherlands\\
Email: \{g.gousios, m.pinzger, a.vandeursen\}@tudelft.nl}
}

\maketitle

\begin{abstract}
Foo
\end{abstract}

\begin{IEEEkeywords}

\end{IEEEkeywords}

\section{Introduction}

Since its appearence in 2005, the Git version control system has 
revolutionized the way distributed software development has been
carried out. 

A unique characteristic of Github as a code hosting site is the fact that it
allows any user to fork any public repository.  The clone creates a public
project that belongs to user that cloned it, so the user can modify the
repository without being part of the development team. What is more important is
that Github automates the selective contribution of commits from the clone to
the source, in a mechanism called pull request.  Pull requests are not unique to
Github; in fact, the Git software distribution includes the
\textsf{git-request-pull} utility which provides the same functionality at the
command line. Github pull
requests\footnote{\url{https://github.com/blog/712-pull-requests-2-0}} improve
Git pull requests by introducing code reviews and integration with its project
forking and issue management facilities, effectively lowering the entry barrier
for casual contributions. Combined, cloning and pull requests create a new
development model, where changes are pushed to the project maintainers and go
through code review by the community before being integrated. 

In this paper, we provide a quantitative examination of the way
pull requests work on Github. Specifically, we examine 50 large Ruby and 
Java projects and identify common factors that affect pull request
handling and merge time.

\section{Collaborative Development models with Git}

Collabora

\begin{description}

  \item[Shared repository] Developers share a common repository, with read and
    write permissions. To work on it, they clone it locally and 

  \item[Pull requests]

  \item[Mixed]

\end{description}

\section{Research Questions}

\begin{itemize}

  \item How are pull requests are being used

  \item Which factors affect 

\end{itemize}

\section{Data collection}

To answer the research questions, we used the GHTorrent program to collect

\section{Results}


\section{Related Work}

Global software development
\section{Conclusions}


\end{document}
